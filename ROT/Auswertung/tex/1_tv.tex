\section{Teilversuch 1: "Gewichtsverlust" des Maxwellschen Rads}
    Das Maxwellsche-Rad hat während des Auf- und Abrollens (außer Umkehrpunkt) immer eine nach unten gerichtete Beschleunigung. Diese Beschleunigung wirkt in derselben Richtung wie die Erdbeschleunigung. Im Vergleich zum Ruhezustand zieht das Maxwellsche-Rad deswegen nicht so viel auf dem Faden. Das folgt aus der sichtbaren Beschleunigung während des Auf- und Abrollens und dem Newtonsche 3. Gesetz. Die gemessene Kraft ist folglich geringer und die Waage hebt sich ab. 

    Am tiefsten Punkt kehrt das Rad um. Das führt zu einer Impulsänderung in einem kurzen Zeitintervall von einem maximalen Impuls nach unten zu einem maximalen Impuls nach oben. Mit $F = \dv{p}{t}$ versteht man diese Impulsänderung als eine große Kraft, die auf den Faden zieht. Diese Kraft wurde als einen Zuck auf der Waage beobachtet. 

    Die Gewichtskraft und die Seilkraft wirken immer auf dem Rad. Da das Rad ab und aufrollt, sind diese zwei Kräfte nicht im Gleichgewicht und es gibt eine resultierende externe Kraft, die den Translationsimpuls des Rads ändert. Da die Seilkraft nicht am Schwerpunkt des Rads greift, gibt es auch ein Drehmoment bezüglich der Rotationsachse des Rads, die den Drehimpuls ändert. 

    Es gibt deshalb zu jedem Zeitpunkt eine externe Kraft bzw. Drehmoment auf dem Rad, und folglich ist das Rad kein Inertialsystem. Die Translation- und Drehimpuls bleiben dann auch nicht konstant. 