\section{Teilversuch 4: Identifikation des "falschen" Würfels}
	\subsection{Messreihe}
		Fehler bei jeder Messung $= \SI{0,2}{\second}$

		\textbf{Roter Würfel}
		\begin{multicols}{3}
			\begin{center}
				\begin{tabular}{lr}
				\multicolumn{2}{c}{$M_1$} \\
				\toprule
				Versuch & $T_i / \SI{}{\second}$ \\
				\midrule
				$(5T)_1$ & \SI{56.74}{} \\
				$(5T)_2$ & \SI{56.46}{} \\
				$\overbar{T}$ & \SI{11.32}{} \\
				\bottomrule
				\end{tabular}
			\end{center}
			\begin{center}
				\begin{tabular}{lr}
				\multicolumn{2}{c}{$M_2$} \\
				\toprule
				Versuch & $T_i / \SI{}{\second}$ \\
				\midrule
				$(5T)_1$ & \SI{57.55}{} \\
				$(5T)_2$ & \SI{57.70}{} \\
				$\overbar{T}$ & \SI{11.525}{} \\
				\bottomrule
				\end{tabular}
			\end{center}
			\begin{center}
				\begin{tabular}{lr}
				\multicolumn{2}{c}{$M_3$} \\
				\toprule
				Versuch & $T_i / \SI{}{\second}$ \\
				\midrule
				$(5T)_1$ & \SI{55.78}{} \\
				$(5T)_2$ & \SI{56.01}{} \\
				$\overbar{T}$ & \SI{11.179}{} \\
				\bottomrule
				\end{tabular}
			\end{center}
		\end{multicols}
		\textbf{Grüner Würfel}
		\begin{multicols}{3}
			\begin{center}
				\begin{tabular}{lr}
				\multicolumn{2}{c}{$M_1$} \\
				\toprule
				Versuch & $T_i / \SI{}{\second}$ \\
				\midrule
				$(5T)_1$ & \SI{55,59}{} \\
				$(5T)_2$ & \SI{55,35}{} \\
				$\overbar{T}$ & \SI{11,094}{} \\
				\bottomrule
				\end{tabular}
			\end{center}
			\begin{center}
				\begin{tabular}{lr}
				\multicolumn{2}{c}{$M_2$} \\
				\toprule
				Versuch & $T_i / \SI{}{\second}$ \\
				\midrule
				$(5T)_1$ & \SI{57,27}{} \\
				$(5T)_2$ & \SI{57,02}{} \\
				$\overbar{T}$ & \SI{11,429}{} \\
				\bottomrule
				\end{tabular}
			\end{center}
			\begin{center}
				\begin{tabular}{lr}
				\multicolumn{2}{c}{$M_3$} \\
				\toprule
				Versuch & $T_i / \SI{}{\second}$ \\
				\midrule
				$(5T)_1$ & \SI{56,97}{} \\
				$(5T)_2$ & \SI{56,99}{} \\
				$\overbar{T}$ & \SI{11.396}{} \\
				\bottomrule
				\end{tabular}
			\end{center}
		\end{multicols}
		wobei alle $\overbar{T}$ analog zu \eqref{eqn:mittelT} mit $n = 2$ berechnet wurden.

		Die Rechnung des Fehlers jedes $\overbar{T}$ erfolgt auch analog zu \eqref{eqn:deltaT} wie bei \eqref{eqn:Tfehlerfor2} und ergibt $\Delta \overbar{T} = \SI{0.029}{\second}$

		Die größte Unterschied zwischen den Schwingungsdauer zweier Seite beträgt bei dem roter Würfel die zwischen $M_2$ und $M_3$ ($\SI{11.525}{\second} - \SI{11.179}{\second} = \SI{0.346}{\second}$) und bei dem grüner Würfel die zwischen $M_1$ und $M_2$ ($\SI{11,429}{\second} - \SI{11,094}{\second} = \SI{0.335}{\second}$). Beide Differenz entspricht ein Unterschied viel größer als sogar dreifachen des Fehlerintervalls ($3\Delta T = 3 \times \SI{0.029}{\second} = \SI{0.087}{\second}$).

		Es gibt außerdem keinen signifikanten Unterschied zwischen den beiden Würfeln. Entweder sind beide Würfel "falsch", oder die Fehler wurden wesentlich zu gering eingeschätzt.

