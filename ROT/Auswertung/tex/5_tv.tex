\newpage
\section{Teilversuch 5: Nicht-starrer Körper am Drehpendel}
	Fehler bei jeder Messung $= \SI{0,2}{\second}$

	\begin{center}
		\begin{tabular}{l *{7}{r} | r}
			\multicolumn{8}{l}{Arretierte} \\
			\toprule
			$n$ & $1$ & $2$ & $3$ & $4$ & $5$ & $6$ & $7$ & $\overbar{T_n}$ \\
			\midrule
			$5T_n/\SI{}{\second}$ & \SI{34.20}{} & \SI{34.14}{} & \SI{33.59}{} & \SI{33.59}{} & \SI{34.68}{} & \SI{34.13}{} & \SI{34.16}{} & \SI{6.814}{} \\
			\bottomrule \\
			\multicolumn{8}{l}{Frei Drehbare} \\
			\toprule
			$n$ & $1$ & $2$ & $3$ & $4$ & $5$ & $6$ & $7$ & $\overbar{T_n}$ \\
			\midrule
			$5T_n/\SI{}{\second}$ & \SI{30.08}{} & \SI{30.10}{} & \SI{30.06}{} & \SI{30.98}{} & \SI{30.87}{} & \SI{29.80}{} & \SI{29.84}{} & \SI{6.049}{} \\
			\bottomrule 
		\end{tabular}
	\end{center}
	wobei alle $\overbar{T}$ analog zu \eqref{eqn:mittelT} mit $n = 7$ berechnet wurden.

	Die Rechnung des Fehlers jedes $\overbar{T}$ erfolgt auch analog zu \eqref{eqn:deltaT} und ergibt $\Delta \overbar{T} = (\SI{0.2}{\second})/(5\sqrt{7}) = \SI{0.016}{\second}$.

	Im Vergleich zu dem Fehler hat die Hantel mit arretierten Seitenzylinder eine deutlich größere Schwingungsdauer als die mit frei drehbaren Seitenzylinder. Aus dem gleichen Direktionsmoment des Drahtes folgt, dass die Hantel mit dem arretierten Zylinder ein größeres Trägheitsmoment hat.

	Dies erklärt sich dadurch, dass die freien Zylinder ihre eigene Drehbewegung bezüglich des Laborsystems nicht ändern und folglich liefern sie keinen Beitrag zum Trägheitsmoment des Systems. 

	Man kann hier trotzdem den Steinerschen Satz anwenden, weil das Superpositionsprinzip für die Berechnung des Trägheitsmoments eines aus verschiedenen Teilen zusammengesetztes Objekt gilt. In diesem Fall sind die Schwerpunktachse der zwei Zylinder nicht die gleiche wie die Rotationsachse im Versuch. Deswegen kann man mithilfe des Steinerschen Satz die Zunahme des Trägheitsmoments erklären. 