\newpage
\section{Teilversuch 4: Messung der Resonanzkurve für eine Oberschwingung}
	\subsection{Messreihe}
	Fehler der Frequenzen $\Delta f = \SI{1}{\hertz}$ \\
	Fehler der Spannung $\Delta f = \SI{0.2}{\milli\volt}$

	Raumhintergrund $= \SI{1.1(2)}{\milli\volt}$

	\begin{center}
		\begin{tabular}{l *{10}{r}}
			\toprule
			$f / \si{\hertz}$ & \num{1011} & \num{980} & \num{970} & \num{952} & \num{928} & \num{910} & \num{892} & \num{881} & \num{871} & \num{861} \\
			\midrule
			$U_\text{eff} / \si{\milli\volt}$ & \num{7.33} & \num{7.53} & \num{7.99} & \num{8.85} & \num{10.85} & \num{13.45} & \num{18.33} & \num{23.31} & \num{30.14} & \num{44.53} \\
			\bottomrule
			\toprule
			$f / \si{\hertz}$ & \num{854} & \num{850} & \num{844} & \num{841} & \num{837} & \num{833} & \num{830} & \num{825} & \num{818} & \num{810} \\
			\midrule
			$U_\text{eff} / \si{\milli\volt}$ & \num{64.33} & \num{85.80} & \num{125.53} & \num{149.11} & \num{149.49} & \num{120.65} & \num{96.04} & \num{71.73} & \num{54.43} & \num{41.24} \\
			\bottomrule
			\toprule
			$f / \si{\hertz}$ & \num{801} & \num{790} & \num{780} & \num{760} & \num{742}  \\
			\midrule
			$U_\text{eff} / \si{\milli\volt}$ & \num{34.00} & \num{28.51} & \num{25.66} & \num{22.68} & \num{21.61} \\
			\bottomrule
		\end{tabular}
	\end{center}

	\subsection{Graph}
	Aus der Anleitung des Versuchs MOS lässt die folgende Funktion eine Resonanzkurve beschreiben:
	\begin{equation}
		\hat{x} = \frac{\omega_0^2}{\sqrt{\pbrace{\omega_0^2 - \omega^2}^2+\pbrace{2\beta\omega}^2}}~\hat{x}_\text{A} \iff U_\text{eff} = \frac{f_0^2}{\sqrt{\pbrace{f_0^2 - f^2}^2+\pbrace{2\beta f}^2}}~U_\text{A}
	\end{equation}

	Da die Kurvenanpassung mit \gnuplot{} sehr empfindlich auf die Wahl der Startparameter reagiert, müssen die Startparameter sorgfältig ausgewählt. Dafür ist eine grobe Kurveanpassung wie im Versuch MOS mittels Geogebra per Hand durchgeführt:
	\begin{figure}[H]
		\centering
		\includegraphics[width=0.9\textwidth]{geogebra.png}
		\caption{Grobe Kurveanpassung mittels Geogebra}
	\end{figure}

	Der Raumhintergrund und dessen Fehler sind direkt im \gnuplot{} Skript berücksichtigt (Siehe Appendix \ref{appdx:gnuplotTV4}). Der Fehler der Spannung nach dem Abzug des Raumhintergrund wird als $(\SI{0.2}{\milli\volt} + \SI{0.2}{\milli\volt} = \SI{0.4}{\milli\volt})$ angenommen. Der Grund dafür ist, dass wir letztendlich nur eine Messung von der Raumhintergrund haben, obwohl wir bei mehrfache Messungen eher Schwankungen gegen die gleiche Werten bekommen. Daher ist es sicherer bei der Kurvenanpassung einen größeren Fehler zu benutzen.

	\begin{figure}[H]
		\centering
		\input{tv4-plot.tex}
		\caption{\centering Messung der Resonanzkurve des Rohres bei der 2. Oberschwingung\captionbr $\chi^2_{\text{red}} = \num{32.6373} > 1 \implies$ Schlechte Anpassung}
		\label{fig:tvfour-plot}
		\vspace{-1em}
	\end{figure}
	Als Endergebnis erhalten wir:
	\begin{center}
		\begin{tabular}{l r r}
			\toprule
			Variable & Rohausgabe & Gerundet \\
			\midrule
			$f_0$ & \SI{835.89(146)}{\hertz} & \SI{835.9(15)}{\hertz} \\
			$U_\text{A}$ & \SI{2.8473(1317)}{\milli\volt} & \SI{2.85(14)}{\milli\volt} \\
			$\beta$ & \num{7.79(102)} & \num{7.8(11)} \\
			\bottomrule
		\end{tabular}
	\end{center}

	Anstatt einfach die Messpunkte durch eine glatte Kurve zu verbinden, war eine theoretische Kurve auf die Messwerte angepasst. Die Anpassung war leider schlecht, und der Grund dafür liegt vermütlich daran, dass wir Nebeneffekte bzw. andere Faktoren nicht berücksichtigt haben. Es könnte auch sein, dass wir alle Werten im SI Einheiten skalieren sollten, bevor wir die Kurvenanpassung durchführen. 

	Unsere gemessene Resonanzfrequenz $f_0 = \SI{837(1)}{\hertz}$ liegt auch im Fehlerintervall des aus der Kurvenanpassung gefundene $f_0$. Die Werte stimmen also miteinander überein.